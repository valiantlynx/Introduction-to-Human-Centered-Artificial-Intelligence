\documentclass[12pt]{article}
\usepackage[margin=1in]{geometry}
\usepackage{times}

\title{Commentary on NITO's Input for Democratic AI Proposal 232 S}
\author{Name: [Gormery K. Wanjiru] \\ Email: [gormerykombo@gmail.com]}
\date{Word Count: [512]}

\begin{document}

\maketitle

% formal
\subsection{formal}512
This commentary reflects upon the input provided by The Norwegian Society of Engineers and Technologists (NITO) on the Representative Proposal 232 S concerning the future of Artificial Intelligence (AI) governance in Norway. After reviewing NITO's stance, I find myself largely in agreement with their perspective, particularly regarding the unstoppable momentum of AI development, the need for Norway to remain competitive in this technological frontier, and the call for specialized AI regulations rather than general IT frameworks.

Firstly, NITO aptly recognizes AI's irreversible integration into global advancements and its support by tech giants such as Google, Microsoft, and Apple. This international momentum underscores the futility of attempting to halt AI development within Norway's borders. Instead, embracing AI for research and innovative solutions in various fields, including medicine, as suggested by the "Venstre" party, emerges as a pragmatic approach to leveraging AI for national benefit. This viewpoint aligns with the global trend of utilizing AI to tackle future challenges, necessitating active participation in its development to mitigate potential adverse effects.

Furthermore, the emphasis on promoting AI and IT education addresses a critical gap in Norway's readiness for the AI era. The dwindling number of IT engineers poses a significant risk to Norway's capability to partake in and influence the AI domain. Therefore, investing in education and training to build a robust talent pipeline is imperative for sustaining technological competitiveness and innovation.

However, where I diverge from NITO's stance is on the nature of AI-specific regulations. NITO's caution against applying general IT regulations to the nuanced and rapidly evolving field of AI is well-founded. Given AI's unique challenges and the pace of its innovation, Norway should indeed develop flexible, AI-specific guidelines that can adapt to technological advancements. Regulations should foster an environment where experimentation and research are encouraged, distinguishing between intentions and outcomes in their application.

For instance, regulating AI's use in art generation models or image recognition should focus on the application of trained models rather than impeding the training process itself. A balanced approach would allow Norway to contribute to AI's advancement while safeguarding against misuse. This stance acknowledges the strategic importance of AI across various sectors, including defense, healthcare, and scientific research, emphasizing the need for a regulatory framework that supports growth and innovation without placing undue restrictions on technological progress.

Lastly, the point on data regulation highlights a unique opportunity for Norway. Encouraging the use of Norwegian data and AI training within the country through incentives or relaxed regulations for domestic activities can enhance Norway's sovereignty over its AI capabilities and innovations. This approach would not only bolster national security but also foster a localized AI ecosystem that is competitive and self-reliant.

In conclusion, NITO's input on Proposal 232 S presents a thoughtful and forward-looking perspective on AI governance. While agreeing with the majority of their views, I advocate for a more nuanced and flexible regulatory framework that recognizes AI's unique attributes and potential. Such a framework should encourage innovation, responsibly harness AI's capabilities for societal benefit, and position Norway as a leader in ethical and democratic AI development.


% informal
\subsection{informal}392
Diving into NITO's contribution to the big talk on AI Proposal 232 S, here's where I land: mostly nodding along but with a few tweaks here and there. NITO's caught the gist—AI's a train that's left the station, with big names like Google and Microsoft fueling the ride. It's not just about staying in the game; it's about playing it smart. So, when the "Venstre" party talks about using AI to tackle tomorrow's headaches, I'm all ears. It's a call to roll up our sleeves and make sure we're part of shaping AI's impact, especially in fields where it counts like healthcare.

Now, onto the brain drain in IT—NITO's hit the nail on the head. Norway's seeing a dip in IT whizzes, and that's trouble. Without fresh minds in AI and IT, keeping pace is going to be a tall order. So, echoing NITO, beefing up education in these areas is a no-brainer.

But, let's talk regulations. Here's where I veer off a bit. NITO's wary of sticking AI under the broad umbrella of IT rules, and they're not wrong. AI's a different beast, evolving at breakneck speed. Slapping on a one-size-fits-all regulation could stick us in the mud while the rest of the world zooms by. My two cents? Let's craft AI laws that are as dynamic as AI itself. Think guidelines that differentiate between intent and action, especially in areas like art creation and image recognition. For example, outright banning training models on images without explicit permission could hamstring progress. Better to regulate what's done with these models without stifling innovation.

And about data—NITO's on to something when they suggest Norway could do with a leg up in how AI's trained on local data. Making it easier for AI development to happen on home turf, through incentives or lighter regulations for domestic data use, could give Norway an edge. It's not just about playing defense but turning our unique position into an offensive strategy in the AI league.

In wrapping up, NITO's insights on Proposal 232 S are spot on for the most part, blending realism with ambition. Yet, in the spirit of pushing boundaries, I'd argue for a regulatory landscape that's as adaptable and forward-thinking as the AI it seeks to govern. That way, Norway won't just be in the race; we'll be setting the pace.


% original
\subsection{original} 411
ai is something that is already ongoing and there is nothing that the norwegian goverment can do to stop it. this is cause already being used, and surpported by so many and so powerful internation powers. 
from google , microsoft and apple. it useful for research and to get new solutions, 
in everything like medicine etc.we have to make use of that to solve the futures problem just like the "venstre" party 
proposed to solve the futures problems. this would also mean we would be in the development and can minimize harmful effects. 
vi må satse på ki/it utdanning. norways has been getting less and less it ingeniørs. i do agrre that we need an ai regulating power but 
not that AI regulation should happen based on general IT regulation rules. ai is improving at a very fast pace and everyone outside of norway 
wont have this shackle around their feet. we should make rules specific to ai since its so new and something new comes out every month if 
not week. id propose someregulation that is permisive and subject to change. for example direct punishment for doing something for example a 
non-harmful with harmful intent but more lenient if perhaps the same thing was done with research in mind. a good example is for example the 
view on art generation model. if we regulate for a complete ban on using images to train model without permission. that would greatly limit the 
progress of image recognition. this would have consiquences like if another country doesnt have this law their ai could learn to recognize stuff 
in a level unheard of. i dont need to speak much about the uses of image recognition, but its useful for things like military application, medicine, reconecense, deffence, all field sciences, 
etc. this would put norway in hard position. instead i propose something like regulation the use of the trained model. 
that, though harder to regulate, does not impare norway. this is my view to most types of ai in general. most cause even i 
dont know what new types of ai will come out. when it comes to data, its true that ai trained on norwegian data is rare. 
even more so fo ai trained in norway. i think norway should regulate in a way to make this better but not too limiting. 
like perhaps less rules if most of the ai data collection, training and inference happen within norway than oout or even reward/incentives
\end{document}
