\chapter{Main}
My field of study is Artificial Intelligence, 5-year master programme. This mean AI everything around. I have to thing about ai in everything and i do. personally, i use ai to automate anything i find tedious, but in generally, if i don't know i use ai. With this i mean ai gives you the possibility to have a second brain, a place we can brainstorm and bounce off ideas. Thats why the task of UN's SDGs if perfect. 

The reasons why they are goals is cause they are hard. and nobody truly know how to achieve them. Even the ideas and my meaning in this report, are not guaranteed to work. The best i can do is statistically calculate them and even that is based on possibility meaning, its all a sort of a viscious cycle.

But AI is different, its solves problems is a different way. Some of those ways we don't even understand or know why. An AI could solve a problem and humans would start a study to learn how it did it. This is what happened when ai was able to simulate protein folding\cite{protein-folding} where it was earlier suspected and still is suspected to need a  quantum computer to simulate the proteins\cite{Hu2023Quantum}

\section{The Sustainable Development Goals (SDGs)}
the UN's SDGs have had many countries and entities working on them and there has been a lot of progress in them but there is till quite a bit before they are fulfilled. 
Last year the UN released the report concerning the SGD, progress etc\cite{sgds-progress}. According to it, for most of the goals only around half is achieved and in some like the SDG 11 (Sustainable Cities and Communities), it's in the negative and instead slums are on the rise and air pollution is now spreading from urban to rural as well. Here is a visual report from UN that shows how dire it is: \cite{sgds-progress}

If we want to achieve all(most) these goals by 2030 we need to to hear up our efforts. 2030 is only 6 years later. It seems like a long time, but it is only as far as 2018 is in the past.

\section{The Problem}
Most reasons people give to use AI are generally and in a way equal to all countries, for example telemedisin, personalized care, predictive analysis, optimizing a task, etc.

This are all correct but not if our goal is to fulfill UN's goal. We need a more specific solution. First we have to identify the problem. This problem is the same for other goals as well. The main problem is that the divide between the developed countries and the rest is too big. This makes it that having generally solution has a negative impact. For example, the rate of expectance to change is highly dependant on the level of development in most countries. More developed countries have more regulation leading to slower speed to change that less developed countries.

Saying that countries that are developed have the responsibility to help the less fortunate without them getting anything in return doesn't make any sense to me. Most Humans are inherently selfish, and tend to fight whatever is, innate or actively, a threat. Evolution is a clear indication for this as a big reason we are humans is evolving against or to "survive" all adversaries we had. There are multiple studies supporting this, two good books are  Xunzi's Theory of Evil Human Nature and Mencius' Theory of Good Human Nature. The two directly contrast with each other 

From what ive learnt
\begin{quote}[Norway]{me}
    Addresing the bad creates more good than trying to create good.
\end{quote}

I don't know if anyone has said this before but its what i've learnt over the years. For example making a project to fix a personal problem i have or something i just want to use personally, makes me more money that trying to make something to make me more money.

The other saying that comes naturally after this is: 

\begin{quote}[America]{benjamin franklin\cite{Benjamin_Franklin}}
    An ounce of prevention is worth a pound of cure.
\end{quote}

but what does this have to do with SDGs, well self sacrifice is saintly. While it is possible it is impossibly hard for countries to help other countries without them feeling used and that the sovereignty is endangered, basically that they are being used.

I think a better solution is bringing the less developed countries up. That is why i chose these Three.

\subsection{Good Health and Well-being (SDG 3)}
AI can revolutionize healthcare in areas that even 100 doctors could not. With the problem in mind to help less developed countries even a normal Large Language Model(LLM) that is not optimized in medicine could achieve this. 

Access to a doctor is usually scarce to none at all in rural areas. The good thing with Ai is you can have the best ai wherever you are, regardless of infrastructure or development. It wont be training the model but just the inference. This mean its don't really need a lot of power, maybe just solar, wind, windmill, batteries, etc. All of which are easy to get into rural areas.

For example lets let our case be a village in the rural village. Not a town but a village with a couple tens people led by an old wise man. In this village we would have The AI center, a building with a local AI installed with multiple client points. The client point could be voice controlled first, because were assuming low level education and illiterate. The center would be public and must be free. Don't worry there is other ways for the country to make money, like them selling the data they collect of the local language to create better ai(with regulations). 

I know this cause ive lived in such a village. Whenever something happens that you don't know happens, it is usually subject to rumours or complete avoidance. This obviously leads to fear and misinformation. For example where i lived it would be fantastical stuff like devils, spirits etc. This is good to teach kids to avoid them but the problem is it persists to adulthood if they don't get knowledge before then. Then they teach it to their kids and voila misinformation.
Another and more normal use would be this case. The village is 6 hours away from the nearest hospital. Imagine a person gets bit by a black mamba. i played this scenario with chatgpt4(no edits).


\begin{itemize}
    \item \textbf{YOU:} rural cure to black mamba poison. i just got bit. i live in kisumu kenya. there is no hospital neably within 6 hours. i dont want to die help
    \item \textbf{ChatGPT:} It's critical to seek immediate medical attention if you've been bitten by a black mamba. Here are some steps you can take right away:\\

    1. **Remain as calm and still as possible.** Moving a lot can increase your heart rate, spreading the venom faster through your body.\\
    2. **Keep the bitten limb immobilized and lower than the heart.** Use a splint if possible to keep the limb still.\\
    3. **Do not attempt to suck out the venom, cut the bite, or apply a tourniquet.** These methods are ineffective and can worsen the situation.\\
    4. **Call for help immediately.** If you can, contact emergency services, friends, or family who might assist in getting you to medical care faster.\\
    
    Since you are far from medical facilities, it's crucial to try and reach a healthcare provider as soon as possible, even if it means arranging for transport for an emergency evacuation. If you have a local emergency number or can contact local authorities, do so immediately.
\end{itemize}

this has already extended their lives by a couple hours. but its unlikely it will end like this. lets say they panic:

\begin{itemize}
    \item \textbf{YOU:} how long does it take to die if i dont do anything.
    \item \textbf{ChatGPT:} The effects of a black mamba bite can be extremely serious and rapid. Without treatment, symptoms can escalate quickly, potentially leading to severe complications or death within a few hours. The time it takes for symptoms to become fatal can vary based on several factors including the amount of venom injected, the location of the bite, and the physical condition of the individual.\\

    Here's a general timeline of what might happen:\\
    - **Within 10 minutes:** You may start to feel symptoms like tingling, swelling at the bite site, or difficulty speaking.\\
    - **Within 30 minutes to 3 hours:** Symptoms can progress to include muscle weakness, blurred vision, difficulty breathing, and altered consciousness.\\
    
    Immediate treatment with antivenom is essential to counteract the effects of the venom. I strongly urge you to seek any possible help to get to a medical facility or have someone bring antivenom to you as soon as possible. Remember to stay as calm and still as you can while you seek help.
\end{itemize}

Then they think about perhaps they could solve it a little:

\begin{itemize}
    \item \textbf{YOU:} is there hearbs i can mix from around here we have a herbalist who is good with hearbs. and how do they look


    \item \textbf{ChatGPT:} It’s important to understand that there are no scientifically proven herbal remedies that can effectively neutralize black mamba venom. The most effective treatment is the administration of the appropriate antivenom. Herbal treatments or traditional methods have not been verified through research to be effective against such potent snake venoms and relying on them instead of seeking medical help can be dangerous.\\

    However, since you are in a remote location and have mentioned that medical help is hours away, here are a few common herbs used traditionally in some regions for general snake bites, but remember, they are not a substitute for medical treatment:\\
    
    Neem (Azadirachta indica): Often used in traditional medicine for its anti-inflammatory properties. The leaves are typically chewed or made into a poultice.\\
    Turmeric (Curcuma longa): Known for its anti-inflammatory properties. It can be applied as a paste on the wound but it is not an antidote for the venom.\\
    Echinacea: Sometimes used to enhance the immune system in general cases, not specifically for snake bites.\\
    Garlic: Applied on the skin for its supposed anti-inflammatory effects.\\
    These herbs are commonly found in many regions, but their efficacy against snake venom, particularly one as potent as that of the black mamba, is not supported by medical science.
    \\
    What you can do now:
    
    Contact the herbalist for immediate assistance in managing symptoms, but you should continue trying to get to a hospital or get someone to bring antivenom to your location as soon as possible.
    If possible, describe your symptoms over the phone to a medical professional who might guide you on immediate steps to take while you arrange for evacuation or delivery of antivenom.
    Your best chance of survival is still reaching a medical facility where the appropriate antivenom can be administered. Stay calm and immobilize the affected area, and avoid exertion to slow the spread of venom.
\end{itemize}


Now I cant verify if this is true but just this knowledge can save multiple lives. You'll be shocked how often this happens.

Even in developed countries people die from sucking poison, cause they saw a movie or read a fantasy book.

Now speaking of developed countries, as you've probably guessed, the same level basic LLM could save lives. imagine them having personal ai in their phone. In the same situation They would be saved just following the first instruction and waiting for the rescue helicopter to come. if they are bit bu a black mamba. The developed country would probably deploy a rescue helicopter. and in less than 30 minutes, hopefully 10 minutes, they would be saved.


\subsection{Quality Education (SDG 4)}
In the same situations as before, rural countries would have access to knowledge for free. Knowledge is always good it gives you peace of mind and make your life easier.
Kids in rural areas could be taught by an ai that is as 80 percent of a field specialist. Yes, there is hallucinations sometimes, but one occasional false information for 100 correct ones is a sacrifice i would take any day.

There is also the bonus the country can calibrate the ai to teach information that needs to be national. For example is there a shortage og data engineers. Make the AI teach up more engineers. Do you need more doctors, lawyers, mathematicians, you can get anybody. 

We have no Idea if the next einstein is in some farm somewhere.

I developed countries, education is almost no issue. Thats why i don't think we need to give this too mush attention as the developed countries. Remember that the problem is that the difference between development in countries is(partly) what causes trying to make worldwide changes impossibly hard. This doesn't mean we can just ignore them. AI can be used where its needed. AI could increase accessability of education to even kids who couldn't get it before. perhaps, the local school is not built with physical accessability in mind, leading to disabled kids not having optimal education. They could also try to get a personalized AI for each student. Kids have hobbies that are not taught at school. for example Programming was mine but at school i was learning basic stuff like biology etc. while it is important the problem is they dint offer or even ever mention the possibility of other stuff like. I also had a friend who has a hobby in cars, another chess, etc. all this were never addressed as the school taught the curriculum.

you might ask why did i not recommend personal ai to undeveloped countries? we it not that i don't but, undeveloped countries are undeveloped for a reason. usually money. so the basic ai and education would be more than enough and faster to improve the education of the nation it turn creating more people for the cows to milk more money from. then use the money to take better care of the cows. While I jest, this analogy works:
I have malnourished cows and no money for feed, the best thing i could do is instead of buying few high quality feed for the money i have, i would but a lot of normal quality feed and fatten up my cows.
But if i Have normal healthy cows with enough feed, the best thing would be to buy high quality feed and optimize my cows for premium milk.
\subsection{Sustainable Cities and Communities (SDG 11)}
This is one of the goals that is growing negatively. Slums are on the rise and air pollution is now not a problem of urban cities but it going to rural areas as well. But i think there is still hope. specifically in the in the congestion but also the slum issue.

The SDGs are all interconnected and solving one cause a chain reaction that solves others.  The slum issue is usually caused cause the influx of population from rural areas is faster than a city can handle. To solve it, you could either find a way for the city to handle it or reduce the influx. the earlier two goals address reducing the influx of people into the city. With better care of rural areas and improvements there, people would have less need to move to cities. people move usually because of better opportunity, but also large ly because of development in cities. They have a greater chance of getting better healthcare, education and money. If we do the earlier simple steps, the only main reasons left would be that people would move to cities for the connection and money. 

This would make the cities into essentially hubs(what they are best at being). This is cause you don't live where connections and money is, its just a place to meet up. More people would live and stay in rural areas but still not impairing movement to cities. all that would be left is cities to make better infrastructure once they have the money.

infrastructure leads us to the problem of air pollution. This being largely contributed by traffic. less people in cities has been shown to not really affect traffic. Traffic can be created by even just 10 cars driving with a lot of space between them. Unless the cars are all being driven by Auto pilot(hehe... AI), there will always be traffic. So this seems hopeless then right? Absolutely Not. The way I see it traffic is the best testing ground for AI. Drivers are taught to drive while expecting the unexpected, this mean whatever AI we put to for example control the traffic lights, however imperfect it will probably be, we can just put it there. Using AI always comes with a risk, but in traffic, AI's risk is mitigated by people. everyone in the city will help the ai get better and the AI will help the people in the city avoid traffic. 

YOu might see some problem with what im saying like for example, would people just stay calm knowing the ai is controlling the traffic lights. But once you think about it that works out for the better. The more problem people have in traffic, the better the ai will learn and recalibrate them to mitigate it.

lets say there is a person who is really against listening to the AI and decides to drive irregularly. well, the ai would learn from that irregularity and perhaps learn to detect it. then maybe find out that perhaps slowing down the traffic that will meet with that irregular driver would solve it, by letting him go first. This is just one of the example that i can think of but who know what a neural network will come up with. That is their job and they have surprised up time and again.