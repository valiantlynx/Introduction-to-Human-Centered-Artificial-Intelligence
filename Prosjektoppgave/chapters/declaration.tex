

\large{\bf{Obligatorisk gruppeerklæring}} \\

{\small \hbadness=10000
Den enkelte student er selv ansvarlig for å sette seg inn i hva som er lovlige hjelpemidler, retningslinjer for bruk av disse og regler om kildebruk. Erklæringen skal bevisstgjøre studentene på deres ansvar og hvilke konsekvenser fusk kan medføre. Manglende erklæring fritar ikke studentene fra sitt ansvar.\\

\begin{center}
\begin{tabular}{ |p{1cm}|p{11.5cm}|p{1cm}|}
 \hline  
 
 1. & Vi erklærer herved at vår besvarelse er vårt eget arbeid, og at vi ikke har brukt andre kilder eller har mottatt annen hjelp enn det som er nevnt i besvarelsen. & Ja \\
 \hline
 2. & \textbf{Vi erklærer videre at denne besvarelsen:}
 \begin{itemize}
    \item Ikke har vært brukt til annen eksamen ved annen avdeling/universitet/høgskole innenlands eller utenlands.
    \item Ikke refererer til andres arbeid uten at det er oppgitt.
    \item Ikke refererer til eget tidligere arbeid uten at det er oppgitt.
    \item Har alle referansene oppgitt i litteraturlisten.
    \item Ikke er en kopi, duplikat eller avskrift av andres arbeid eller besvarelse.
 \end{itemize}& Ja \\
 \hline
 3. & Vi er kjent med at brudd på ovennevnte er å betrakte som fusk og kan medføre annullering av eksamen og utestengelse fra universiteter og høgskoler i Norge, jf. Universitets- og høgskoleloven §§4-7 og 4-8 og Forskrift om eksamen §§ 31.
 & Ja \\
 \hline
 4. & Vi er kjent med at alle innleverte oppgaver kan bli plagiatkontrollert.
 & Ja \\
 \hline
 5. & Vi er kjent med at Universitetet i Agder vil behandle alle saker hvor det forligger mistanke om fusk etter høgskolens retningslinjer for behandling av saker om fusk.
 & Ja \\
 \hline
 6. & Vi har satt oss inn i regler og retningslinjer i bruk av kilder og referanser på biblioteket sine nettsider.
 & Ja \\
 \hline
 7. & Vi har i flertall blitt enige om at innsatsen innad i gruppen er merkbart forskjellig og ønsker dermed å vurderes individuelt.
Ordinært vurderes alle deltakere i prosjektet samlet.
 & Ja \\
 \hline
\end{tabular}
\end{center}}

\bigskip

\large{\bf{Publiseringsavtale}} \\

{\small \hbadness=10000 Fullmakt til elektronisk publisering av oppgaven
Forfatter(ne) har opphavsrett til oppgaven. Det betyr blant annet enerett til å gjøre verket tilgjengelig for allmennheten (Åndsverkloven. §2).
\\
Oppgaver som er unntatt offentlighet eller taushetsbelagt/konfidensiell vil ikke bli publisert.\\

\begin{center}
\begin{tabular}{ |p{13cm}|p{1cm}|}
\hline
Vi gir herved Universitetet i Agder en vederlagsfri rett til å gjøre oppgaven tilgjengelig for elektronisk publisering: & Ja \\
\hline
Er oppgaven båndlagt (konfidensiell)? & Nei \\
\hline
Er oppgaven unntatt offentlighet? & 
Nei \\
\hline
\end{tabular}
\end{center}
}